\section*{Introduction}
\addcontentsline{toc}{section}{Introduction}
\markboth{Introduction}{Introduction}

\tab L'année dernière, le projet Hermès a été développé, visant la création d'une plateforme motorisée mobile permettant de guider des personnes dans le forum de Télécom SudParis. Cette année, l'objectif a été de lui ajouter un dispositif de détection afin de lui permettre de se repérer au sein de son environnement et de détecter d'éventuels obstacles. Pour cela, nous avons proposé l'utilisation d'un LiDAR (\textit{Light Detecion And Ranging}), système permettant la récupération de nuages de points autour du système. Malheureusement, la plateforme Hermès n'étant pas en état de marche, nous nous sommes rabattu sur le robot du club INTech qui a participé avec notre système à la Coupe de France de Robotique afin de réaliser nos travaux qui seront détaillés dans ce dossier.

\tab Les dispositifs LiDAR sont très utilisés dans le monde de la robotique, notamment par Boston Dynamics ou encore Nexter. Ils permettent de cartographier le monde autour du robot afin d'interagir ou d'éviter des obstacles. Ils sont également omniprésents dans les projets de voiture autonome comme les récentes Tesla. Cependant, les données brutes d'un LiDAR sont très basiques, il faut alors tout un traitement en aval par des algorithmes qui formeront ce que nous pouvons appeler un véritable \textit{capteur logiciel}.

\tab Le problème posé est alors celui du traitement des données brutes d'un LiDAR dans l'objectif d'une détection d'obstacles et de localisation (un robot adverse par exemple dans ce cas précis) permettant à un algorithme de recherche de chemin de pouvoir oeuvrer. Cette recherche de chemin associé à notre LiDAR a fait l’œuvre d'un second projet Cassiopée, Hermès - Pathfinding.